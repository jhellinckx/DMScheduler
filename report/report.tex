\documentclass{scrartcl}

\usepackage[T1]{fontenc}
\usepackage[utf8]{inputenc}
\usepackage{fourier}
\usepackage[english]{babel}
\usepackage{fullpage}
\usepackage{graphicx}
\usepackage{hyperref}
\usepackage{url}
\usepackage{listings}
\usepackage{color}
\usepackage[toc,page]{appendix}

\title{INFO-F404 - Real-Time Operating Systems\\Project 1: Global vs Partitioned DM}
\author{Anthony Caccia \and J\'{e}r\^{o}me Hellinckx}
\date{\today}

\begin{document}
\maketitle

\section{Introduction}
This project consist in studying performances of \emph{Deadline-monotonic scheduling} (DM) algorithm on systems with two different strategies: global and partitioned with best fit. Systems have $n$ periodic, asynchronous and independant tasks $\tau$ with constrained deadlines on a multiprocessor systems.

DM is a \emph{Fixed Task Priority Scheduler} (FTP): each tasks are ranked determinstically by their deadlines: the lower the relative deadline, the higher the priority.

Partitioned strategy aims to find a packing where each task cannot migrate from one processor to the other. In this implementation, the rule to assign tasks is \emph{Best-Fit}: tasks look for remaining utilisation on each processor and is assigned to the processor with the minimum remaining.

Global strategy, on the other hand, permits to tasks to migrate between processors during their lifetime: they can then start their execution on a processor and resume on another one.

\section{Code description}
Three executables were asked in order to complete this project:
\begin{enumerate}
  \item a simulator, taking as list of tasks, a number of processor and a strategy and simulate the system on the interval $I = [0; O_{max} + 2 * P]$;
  \item a generator, which creates a list of $n$ tasks with an utilisation $u$;
  \item a study program, to test DM's performances.
\end{enumerate}

\subsection{Simulator}

\subsection{Generator}

\subsection{Study}

\section{Encoutered problems}

\subsection{Simulator}

\subsection{Generator}

\subsection{Study}

\section{Comparison tests specification}

\section{Results}

\end{document}
